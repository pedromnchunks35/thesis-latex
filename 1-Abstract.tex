This thesis investigates the benefits of adopting private blockchain technology, among other recent innovations, within the health sector. Even though organizations have been concerned about security for a long time, there is a trade-off between security and the exchange of data between entities across the network. The primary attribute of blockchain is thus ensured to provide better integrity, security, and flexibility in sharing information securely across a group or consortium of organizations.

During this project two main use cases where created: integration of \textbf{Hyperledger Fabric} with \textbf{IPFS}  and optimization of management of blockchain infrastructure by means of contemporary technologies. In further sections, each of use cases is deeply elaborated to give an extended look at their practical usage and results.

The first use case would assume the benefits of using \textbf{Hyperledger Fabric} in combination with \textbf{IPFS} for secure file storage, where a private network of \textbf{IPFS} would store the files and record hashes on the blockchain for integrity.In the second use case on the other hand, there is the design and benchmarking of several network infrastructures of blockchain seeking an optimal configuration. Besides that, there is a discussion about the development of UI tool for monitoring and analysis of the status of the network. To this precise use case,more work is expected in the future to enhance this use case further, to refine even more its robustness, operability and maintainability.

This work will demonstrate such use cases and illustrate how blockchain can enhance data security and operational efficiency in healthcare while resolving challenges in large-scale implementations.

% Example keywords section with indentation
\vspace{1cm} % Add space before the keywords section
\textbf{Keywords:} \parbox[t]{0.6\textwidth}{\raggedright Blockchain, Blockchain in Healthcare, Healthcare Systems, Hyperledger Fabric, Kubernetes;}
\medskip

\pagebreak