\subsection{Context and Motivation}
The healthcare industry is in the landscape of a radical modification, dragged by indulging advances in technology, converting expectations from patients and a highly complex regulatory environment. From digital health solutions to electronic medical records, telemedicine, and connected devices, the proliferation of data is happening in healthcare at rates never seen previously. This explosion of data opens up new opportunities and creates challenges. While on one hand, access to huge volumes of patient information has the potential to transform diagnosis, treatment and preventive care - all to improve patient treatment - this could be problematic because the increasing complexity of healthcare systems coupled with rising concerns on data privacy and security creates considerable barriers to benefits realization.

The healthcare organization, therefore, finds itself between two poles of extremes: driving innovation through data on one hand, and on the other hand, ethical and legal imperatives of maintaining patient privacy. This gets even more imperative as most healthcare systems across the world are fragmented in nature, encompassing a broad range of stakeholders that include hospitals, laboratories, insurance companies, regulatory bodies, and pharmaceutical firms among others, all of whom will have to share closely guarded information. Ensuring that the data remains accurate, secure, and accessible only to authorized parties stands as one of the most formidable issues associated with healthcare today.

Traditional medical data management systems are prone to inefficiencies and vulnerabilities. Centralized databases remain at the mercy of even basic cyber-attacks, data breaches and manipulation from internal sources. Moreover, many old health management systems face operational difficulties, such as non-interoperability, which complicates the free flow of information between different agencies. The above limitations signify the urgent requirement for more secure, flexible, and trustworthy systems for managing healthcare data.

But working amongst all these critical challenges, the blockchain technologies have inspired confidence. What was ostensibly conceived as the underpinning technology for cryptocurrencies like \textbf{Bitcoin}, blockchain itself-in core concepts of decentralization, transparency, immutability  and cryptographic security-has kindled interests in industries way beyond finance. Particularly in healthcare, blockchain holds real promise to offer an indestructibly secure way of recording transactions and managing data where trust and security are paramount. Blockchain can provide a decentralized solution that removes some of the associated risks of using centralized databases by distributing the data across a network of nodes, none of which has unilateral control over the entire system.

Within the sphere of blockchain frameworks, \textbf{Hyperledger Fabric} shows up and shines as it is considered one of the most convenient for the healthcare sector. \textbf{Hlf} is described as an open-source project covered and performs as a permissioned blockchain framework for enterprise usage, thereby perfect as well for this thesis but this will be covered more ahead. Unlike public Blockchains, which allow everyone to participate, \textbf{Hyperledger Fabric} is designed for participation in environments where participants are known and trusted. This is an important permissioned model necessary for healthcare, where regulatory compliance-like \textbf{GDPR} and \textbf{HIPAA}, along with patient privacy is a non-negotiable requirement. Another very important aspect of the framework is the modular architecture, which provides great flexibility because an organization can adapt the system to suit their needs in any area of concern that it deems fit-whether record keeping, tracing drugs, or settling claims.

The thesis is aimed at discussing the application of Hyperledger Fabric as a transformative technology to approach the most important challenges faced by healthcare systems. The key focal area in this research is how Hyperledger Fabric can be mobilized to advance the qualities of data integrity, trust, flexibility and coordination in the healthcare ecosystem. Therefore with this, we will answer the very research questions which are "How can a private blockchain network be utilized to improve data integrity,security,flexibility and collaboration within the healthcare ecosystem?" and "What kind of infrastructure design is necessary to support a blockchain solution in such a vast and complex environment as healthcare?"  By implementing blockchain through a distributed ledger, cryptographic algorithms and consensus mechanisms, Hyperledger Fabric is poised for Layer 1 utilization in building a novel model of healthcare data management that engenders trusted expectation among stakeholders in the assurance of sensitive healthcare information being controlled against unauthorized access and tamper-proof.

We will start with data integrity in healthcare, which at its core wants correct and tamper-proof records to form the very bedrock of diagnosis, therapy, and even billing. The immutability ledger in \textbf{Hyperledger Fabric}-finally cryptographically linked and unable to be altered-offers a big advantage over traditional databases. The feature will ensure that the data can't be tampered with and will provide a clear, transparent audit trail to verify if any information at any given time is genuine. Letting patient data, medical history or prescription be recorded in a secure way, \textbf{Hyperledger Fabric} may become the technology capable of excluding frauds of data manipulation and unauthorized access altogether. The other focus is data flexibility and interoperability, which have been a nightmare in health care for quite some time. Often, there are different systems within different health organizations that are incompatible with one another and usually lead to data silos, which cannot easily facilitate smooth information exchange. The modular design of \textbf{Hyperledger Fabric} allows innovation in health care to build a blockchain-based solution that interfaces with existing systems and is agile to adapt to future technologies.

This flexibility not only greatly eases migration from older systems but also promotes innovation in that new functions can easily be added as the needs of the industry develop.

More importantly, this research will go a long way in addressing the need for collaboration among healthcare stakeholders. Since the data is scattered across hospitals, insurance companies, pharmaceutical firms and regulators, the ability to share information in a secure manner has never been so critical.

Permissioned network and granular access control mechanisms are provided by \textbf{Hyperledger Fabric} to enable wide collaboration among health organizations and ensure that sensitive data is inaccessible to unauthorized parties. Preceding with such control provides a trusted setting where stakeholders are free to share their information, knowing it will help achieve greater advances more quickly, improve patient care, and give way to smoother processes. Beyond both of these key themes, consideration is given to the specific use cases of \textbf{Hyperledger Fabric} in healthcare: supply chain management, identification of patients, and clinical trials. Each of these has unique problems that blockchain technology can help solve.

For instance, origin and journey tracing in supply chain management will detect the risk of counterfeit drugs reaching the market. In identity verification, \textbf{Hyperledger Fabric} will help the patients to be in control of their health records and grant access to specific providers with permission instead, hence advancing privacy while reducing the risk of identity theft.

In this respect, the thesis will discuss the technical deployment of \textbf{Hyperledger Fabric} by utilizing state-of-the-art technologies such as \textbf{Kubernetes}, mainly in on-premise environments where healthcare organizations would wish to have full control of their infrastructure. Thus, using \textbf{Kubernetes} for the deployment of a \textbf{Hyperledger Fabric} will provide the ability to assess the scalability, security and performance of this deployment and then compare it against the conventional systems in operation within the context of today's healthcare. The technical assessment will also include an analysis of the advantages and disadvantages of such deployment, thereby offering practical insights into any healthcare organizational considerations towards blockchain implementation. Finally, the research will try to provide a holistic pathway on the expected work regarding the future of the second use case project, considering potential risks and pitfalls that might arise, areas of investigation, and more. While huge, the benefits of blockchain are not devoid of problems in application, including most especially, those related to governance, regulatory compliance, and integration into the existing healthcare system. As such, this thesis tries to give equal attention to considerations in an effort to balance the thesis as regards to the role \textbf{Hyperledger Fabric} can play in transforming healthcare. To sum up, the present thesis will critically assess the magnitude of transformational impact Hyperledger Fabric may have on the healthcare industry through the in-depth review of its applications and respective technical and strategic consequences. As we forge ahead into a future characterized by data-driven innovation, collaboration, and trust, blockchains could provide that much-needed tool in reshaping the health environment to ensure not only the secure but also efficient and ethical management of patient data.

Consequent upon the rapid development and increase in health data volume, security, and privacy concerns, and efficiency, the healthcare industry currently finds itself at an extremely critical juncture. As health care providers increasingly adopt digital health solutions-such as electronic medical records, telemedicine platforms, and connected devices-the overall volume of health care data resources is developing at an exponential rate. This data outrun instigates a vast number of  opportunities to improving systems that help in taking care of patients, but it also compiles a serious deck of challenges in data management, security, and sharing. Perhaps the most relevant of this deck is the concept of data integrity, which remains fragile in healthcare systems as in any worldwide realm.

These inaccuracies, unauthorized accesses and data changes may critically lower patient safety and impact the quality of care. Traditional data management infrastructures, often centralized in architecture, have proven very vulnerable to many forms of data breaches, cyberattacks and insider threats. What could represent an example of this shameful situations is, in the healthcare organizations sphere, this being the biggest target for ransomware attacks, where many have resulted in an unfortunate financial and reputational disasters. In 2021 solo, more than 40 million health record breaches were 
declared in the United States. Furthermore, fragmented health systems disappoint further by creating silos of data where different organizations have incompatible systems, creating barriers to collaboration and limiting interoperability, furthering vulnerabilities. Other than security challenges, demands for efficient, flexible and interoperable systems have never been more urgent.

Health care organizations struggle with outdated, legacy systems that do not permit seamless integration across providers, insurers, and regulatory agencies. All these latter systems are frequently incompatible with each other, making it very difficult to facilitate the real exchange of critical health data in real time. This lack of coordination obstructs innovation and operating efficiency, delaying treatment, inflating health care costs, and adversely impacting patient outcomes.

Moreover, the healthcare system relies heavily on trusts: both in data shared between stakeholders and in security around that data. Patients, providers, insurance companies, and regulators all need to know that when they collaborate, their information is exchanged correctly, securely and in a tamper-evident manner. Obviously, many systems today do not ensure this level of trust, with all the associated benefits of collaboration and cooperation that would enable improved patient outcomes. Abstracted, to engage against regulatory frameworks such as \textbf{HIPAA} and \textbf{GDPR} and gather new solutions and have in mind restrictions related to the usage of traditional providing of data or even cloud storing that may suffer from scalability,  there comes the urgent of improving the current technology landscape presented in the healthcare facilities.

\subsection{Finality and Objectives}

If putted into deep thinking, in the sphere of the modern new world, blockchain technology has loom as an auspicious solution to the given number of challenges faced by the healthcare industry. It's enormous traits can range from cryptographic security, transparency and immutability and ,by retaining such characteristics, it also seeks influence in securing and managing sensitive data. It aims to safeguard the gathering of data across a networking of nodes, making it tamper-proof and erasing the traditional single points of failure. In another deck of terms, blockchain can enable smooth data allocation between heterogeneous entities and yet making certain that only conceded parties have privileges to specific and maybe confidential pieces of information.

Within the vast array of blockchain frameworks, \textbf{Hyperledger Fabric} looms as particularly well-suited to addressing the specific needs of the healthcare environment. It provides the flexibility, scalability and security that serve as the genesis block for enterprise-level applications. Unlike public blockchains such as \textbf{Bitcoin} or \textbf{Ethereum}, \textbf{Hyperledger Fabric} permissioned model allows more control over access to the network, ensuring that only trusted participants are allowed to validate and access data. This is a crucial feature in healthcare, where strict privacy regulations and the need for secure, authorized data sharing are very important, not speaking about the need of on premise implementations which are also very possible with this framework.

\textbf{Hyperledger Fabric's} modular architecture also enables organizations to customize their networks to meet specific needs. It can be used for medical record management, pharmaceutical supply chain tracking, or patient identity verification, the platform can be adapted to a variety of use cases. This flexibility not only facilitates innovation but also supports the integration of blockchain technology with existing legacy systems which reduces the friction associated with technological adoption in healthcare settings.

The research and project is motivated by the growing recognition of the transformative potential of blockchain and also by the major other main technologies used currently in the market. With frequency, lots of people discuss the matter of decentralization, but lack on the sphere of trying to solve the problem in hands. When there is a depth relation within the creation of an conjunction between centralized and more decentralized technologies, we get the means to empower healthcare institutions to face their real problems. While blockchain has already been explored in sectors such as finance and supply chain management, it's application in the healthcare sector remain relatively fledgling. Nevertheless, early trials and implementations have been showing promising in aprimoring data security, interoperability and collaboration between multiple organizations or even stakeholders.

The reliance from the healthcare on trust makes it an ideal candidate for the blockchain landscape, even more if there is careful thinking about the need of securing and verifying sensitive patient data. However, deploying such complex system in the healthcare context does not cease to be very challenging and demanding. A lot of possible barriers to achieve this can be the Technical, regulatory, and operational spheres of the sector. Scalability is also another dual knife problem, and remains as so in large healthcare environments with enormous and vast amounts of information and high transaction volumes. This is a huge concern in both worlds since, within a blockchain there is the need of careful thinking such that there is the concern regarding the degree of consensus which delays requests out. Furthermore, the integration of blockchain with legacy systems and compliance with existing regulatory frameworks must be thoroughly evaluated.

This thesis is driven by a desire to explore these challenges and opportunities in depth. Specifically, it seeks to understand how Hyperledger Fabric can be designed, implemented, and deployed to address the critical needs of healthcare organizations. By investigating the ways of deploying a private blockchain network within healthcare systems.

This research aims to answer the following key questions:
\begin{enumerate}
\item \textbf{"How can a private blockchain network be utilized to improve data integrity, security, flexibility, and collaboration within the healthcare ecosystem?"}
\item \textbf{"What kind of infrastructure design is necessary to support a blockchain solution in such a vast and complex environment as healthcare?"}
\end{enumerate}

This questions are imposed, so that it becomes more evident in a qualitative way the achievement of the result in mind. Further assumptions about actual quantitative goals must be ensured as the thesis goes along but answering this questions is the basis to come up with a suited solution in practical but also in theoretical knowledge terms.

As obvious as it may be, every research must have it's own objectives and this one could not be different, specially this one where there is the possibility to gather multiple objectives due to the countless numbers of opportunities and challenges that could be faced regarding this procedure of recursion to the blockchain. This becomes even more clear according to the area of investigation which is healthcare, a institution that depending of the department becomes a high level risk sector to face, not only because of that but also mainly because of it's many problems, it's public/private nature that is always changing, the regulations required to be apart of the procedures, the continuous innovation, the fact that is researched by a vast number of different areas, the number of stakeholders involved, the number of organizations involved and the change resistance. With this in mind, the objectives covered will be the most general possible and they will aim to solve the majority of problems that the project may face and with this there are a lot of risk as well but those will be also described further in this work. The objectives that are intended to be reached in the realm of this work are the following:

\paragraph{1. Answer Research Questions}\mbox{}\\
\textbf{Answer Research Questions} is the basis of the project, if answered this question will generate practical and also theoretical insights that will foster innovation. The questions mentioned here, are the ones that were acknowledged in the section 3, being those \textbf{”How can a private blockchain network be utilized to improve data integrity, security,
flexibility, and collaboration within the healthcare ecosystem?”} (question 1). And \textbf{”What kind of infrastructure design is necessary to support a blockchain solution in such a
vast and complex environment as healthcare?”} (question 2), respectively. As it will be shown further in this thesis, to gather the information to answer such answers, there will be mentioned the existence of 2 major use cases: One works more as a extension, altought inspected firstly and the second one, which is conducted using \textbf{DSR}(Design Science Research), is what is required as genesis block for creating use cases that can then me extended using the first use case. In other terms, the first one is mainly theoretical  and the second one try's to solve a real world problem within the healthcare realm. Answering those questions, not only is important for the thesis itself but also for the healthcare sphere and other contexts that could adopt the same logic to solve their problems, being essential answering them as such occurs where both practical and theoretical inceptions are very important. Making the second project more appealing due to having proven results to achieve a given solution.
Despite this being the main objective, inside of the second use case there are others that are more quantitative, thereby more concrete. Some can be used as metrics in other contexts, others may not but the importance of these questions remains important altought not that much specific when it comes to the refereed metrics and even the requirements disposed in the use case section.

Both questions are answered by using the conjunction of both research prior use case and by delving into the use case itself, altought it is believed that the use cases represent better the potentialities of this work because the theory by itself sometimes is not sufficient to cover vast cases, such the case of the sector under study, more concretely, the healthcare environment.

\paragraph{2. Contribute}\mbox{}\\
Another objective that stands out with honor is the honored \textbf{contribution}. Despite this being obvious and also be part of the \textbf{DSR} methodology, it is still important to mention concretely this as an objective. Contribution is very important and avoiding specifying to much in presenting the knowledge produced is also something that should be done with a lot of consideration, since by specifying to much, there is the probable occurrence of not giving to much for other's to work in their own context or situation, therefore making the contribution only suitable in the landscape that is being targeted in this precise research. More concretely, it is founded that creating clear and effective contribution that could be applied to various contexts is another objective that is intended to reach and that is the main reason why there are specific understandings relatively to the use cases presented but also in a general context, such that other researches can feed them with the specific cases but also understand deeper what was reached by using the global view presented in the final of the thesis.

\subsection*{Document Structure}
This dissertation is divided into several chapters: "Introduction,State of Art,Methodologies and Tools, Use Cases,Discussion,Conclusions," and "References.".

\textbf {Introduction}: This chapter is reserved to give a contextualization and expose what is about to be delved. This is accomplished by explaining the context, motivation, finality,objectives and document structure.

\textbf{State of Art}: Within this chapter there is the supply of the most important theoretical concepts that are pretty relevant for the case under scrutiny.

\textbf{Methodologies and Tools}: In this chapter, there is the explanation of DSR and how it is being used currently in general terms, the explanation of SWOT analysis, the expected project plan, a list of risks, and the relevant tools divided by area of concepts (blockchain, microservices, analytics and web development).

\textbf {Use Cases}: On this section, there will be mention of two major use cases that were conceived during the project, standing as the biggest font of knowledge of this work. In each section, objectives, plans, discussions, and even conclusions will be conducted. Also, one of the use cases is more practical, having topics that are more specific to it's sphere, such as implementations, solution comparison, approach, and final architecture.

\textbf{Discussions}: This represents the discussion of the whole project and not a single use case. This is important to reflect over the results obtained as a whole and not only specific to a single use case.

\textbf {Conclusions}: In the conclusions section, there will be plenty of views around final considerations, project limitations, and future work. This will be done in the context of the general project, since every use case already speaks about such things.

\textbf{References}: The references chapter will be around the bibliographical references that were used along the dissertation.