\paragraph{8.1- IPFS and Hyperledger Fabric: Integrity of Data in Healthcare}\mbox{}\\

\textbf{Abstract}: Since the Information of Things (IoT) arrival, one of the main prob-
lems we encounter daily is data breaches and data integrity. Now more than ever,
the expertise needed to develop an attack is decreasing. Developers are creating
software that does the same thing as an expert, requiring less knowledge from the
attacker. Also, ill-intended professionals within the institutions can compromise
and access information without authorization.
Healthcare Information Technologies must be aware of and have a proactive
approach to this problem within each sector. With this in mind, researchers
and developers must propose and study solutions and architectures to store and
query sensitive files and information more securely.
When we think about security, there are more dimensions to consider other
than the technology itself, but it does remove some constraints. This article
presents an architecture that relies on Blockchain through HyperLedger Fabric
and Interplanetary File System (IPFS) to securely host sensitive documents such
as contracts within Healthcare.
\vspace{1cm}

\textbf{Keywords}: Blockchain;Healthcare Industry;Data Integrity;Ipfs;Hyper Ledger
Fabric;Kubernetes;Security;Permissioned Blockchain;Web3;Linux Foundation;

\paragraph{8.2- Scalable and Sustainable Blockchain: Architecting Infrastructure and Developing a Platform for Efficient Management and Exploration}\mbox{}\\

\textbf{Abstract}: The application of blockchain technology in the health and healthcare sector is considered disruptive, thus requiring trust, integrity, and value of data. In addition, this document analyses the potential of permissioned blockchain, more specifically Hyperledger Fabric, to improve the security, integrity, and efficiency of healthcare systems. It examines ways in which this can be done in a manner that has both scalability and sustainability prospects in the long-term. This, by reaching the optimal compromise between efficiency of resources and its ease of use, management and upscaling. The ability to be as lightweight as possible, while being able to quickly expand the infrastructure and seamlessly integrate new functionalities, maintaining operational efficiency. 
The research developed involves a review of the literature, the development of three different blockchain implementations/architectures where the practical assessment of performance metrics is performed. Finally, a blockchain management platform is presented. This was developed to ensure long-term usability and maintenance of blockchain solutions in the healthcare industry.
This work aims to advance the application of blockchain in healthcare, addressing both immediate and long-term needs for security and efficiency.


\textbf{Keywords}: Blockchain; Blockchain in Healthcare; Healthcare Systems; Hyperledger Fabric, Kubernetes;

\newpage