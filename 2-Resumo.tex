No decorrer desta tese, uma investigação sobre os benefícios de adotar tecnologias de blockchain 
privadas será conduzida, assim como outras recentes inovações no contexto do setor da saúde. Posto isto, apesar da
segurança ser uma das prioridades das organizações já há muito tempo, esta continua a ser uma prioridade que põe 
em causa o funcionamento correto das operações afetas ao desenvolvimento das atividades das mesmas, 
mais concretamente aquilo que são transações de dados entre entidades de uma rede homogénea. 
As principais características importantíssimas asseguradas pela blockchain passam pela melhor integridade, 
segurança e flexibilidade em compartilhar informação de forma segura dentro de um grupo ou consórcio de 
organizações.

Durante este projeto, 2 principais casos de uso foram criados: 
\textbf{Hyperledger Fabric} com \textbf{IPFS} e otimização de gerenciamento de uma infraestrutura 
de blockchain com o uso de tecnologias mais padronizadas no mercado de trabalho atual. 
Estes casos de uso serão explicados em mais detalhe nas secções mais adiante, 
demonstrando a sua utilidade prática e também resultados obtidos dos mesmos.

O primeiro caso de uso é mais virado para os benefícios do uso de Hyperledger Fabric 
combinando a tecnologia \textbf{IPFS} para termos também características de segurança no que toca ao 
armazenamento de ficheiros, onde uma rede privada \textbf{IPFS} poderá brilhar no que diz respeito ao 
armazenamento de ficheiros e traqueamento de representações encriptadas dos mesmos na blockchain, para efeitos 
de integridade de dados. Por outro lado, no segundo caso de uso haverá uma explicação daquilo que foi o desenho 
e benchmarking de uma série de infraestruturas de rede de blockchain, procurando aquilo que esperamos ser a 
configuração mais otimizada. Além disso, também haverá uma discussão acerca do desenvolvimento de uma ferramenta 
web para monitorizar e analisar o estado atual da rede em questão. Para este caso de uso em concreto, trabalho 
futuro é esperado e, como tal, ideias de arestas a melhorar são descritas, ideias estas que envolvem aprimorar 
características de robustez, operabilidade e manutenabilidade.

Este trabalho promete demonstrar estes casos de uso e ilustrar em que medida a blockchain pode melhorar 
a segurança de dados e a eficácia da operabilidade dentro do setor da saúde, pelo que é esperado também 
que se resolvam desafios relacionados com implementações de redes em organizações de grande escala.

% Example keywords section with indentation
\vspace{1cm} % Add space before the keywords section
\textbf{Palavras chave:} \parbox[t]{0.6\textwidth}{\raggedright Blockchain, Blockchain in Healthcare, Healthcare Systems, Hyperledger Fabric, Kubernetes;}
\medskip

\pagebreak