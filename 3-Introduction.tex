\subsection{Context and Motivation}
The healthcare industry is suffering a deep transformation, conducted by technology progression where patient expectations and an emerging regulatory environment
are concerns creating both opportunities and headaches.
From telemedicine, connected medical devices to unthinkable volumes of data, challenges become visible.
From one side facing those, potential numerous outcomes can surge such: Revolutionizing diagnosis, treatment and outcome prevention.
On another perspective, besides that amount of complexity there is also the rising concerns regarding data privacy and security. \cite{it-challenges-healthcare}

Organizations struggle when faced with 2 sides of the coin: innovate through data and not abstain ethical and legal responsibilities
to protect patient privacy.
The path for this becomes even narrower if the fragmented nature of healthcare systems is considered.
Characteristic that include a vast array of stakeholders - hospitals, laboratories, insurance companies, regulatory bodies and pharmaceutical firms - where
each of them must share sensitive information.
Making sure the output of the exchange remains accurate, secure and accessible only to authorized parties stands as the most fragile topic in the modern healthcare landscape. \cite{healthcare-data-fragmentation}

Inefficiencies and vulnerabilities are very present in traditional medical data management systems.
Unfortunate situations - centralized database cyberattacks, data breaches and insider manipulation - can occur, due to legacy systems that lack interoperability,
different procedures to share information between heterogeneous parties and human error.
These setbacks underscore the need for more secure, flexible,user-friendly and trustworthy mechanisms for such systems. \cite{legacy-healthcare}

On top of that, the volume of data is rising exponentially.
The cause of such includes new digital solutions - including electronic medical records, telemedicine, connected devices and real time monitoring - bringing with them
concerns over their vulnerability to cyberattacks, data breaches, insider threats and integrity.
The outcome is expected: more data means more ground for casualties but likewise, also instigates for more innovation, creating opportunities to
enhance the services, where data integrity remains as the most crucial characteristic to maintain. \cite{data-volume-increase-healthcare}

Notably, healthcare has become one of the primary targets for threats such - ransomware, phishing, data breaches, ddos (distributed denial of service)
and third party attacks - resulting in severe financial and reputational damage.
Only in 2021, over 40 million health care record breaches were reported in the United States, highly because of the previously mentioned problems: fragmentation,
lack of collaboration and limited interoperability which increase the risk exposure. \cite{data-breaches-2021} \cite{healthcare-privacy}

Finally, in regard to the problems discussed so far, many systems fail to ensure trust, thereby obstructing the collaboration between multiple parties which slows
down innovation and progress.
Achieving this goal, passes a lot through the regulatory demands of frameworks such as HIPPA and GDPR, difficult to attend when there are multiple heterogeneous systems
that can be either premise or cloud-based, fostering even more existing limitations: There is an urgent need to modernize and standardize the technology foundations of
healthcare institutions in a way that extinguishes the difficulty of surpassing bureaucracy. \cite{hippa-gdpr}

Amid these challenges, blockchain technology emerges as a probable solution.
With foundations that came from the still existing and famous Bitcoin - with its core principles
of decentralization, transparency, immutability, cryptography-based ensured security and anonymity - it has been
catching a lot of interest across multiple industries besides finance.
For starters, this technology can serve healthcare by offering a secure and tamper-resistant mechanism
for recording transactions and managing data, ensuring this way trust and confidentiality, potentialized by its
distributed network of nodes, reducing the risks with traditional centralized databases due to the fact that
this same network requires a consensus of integrity for every piece of data that crosses it: No single entity
maintains full control over the system. \cite{btc} \cite{healthcare-data-breaches}

Within the blockchain landscape, one framework has become the standardized one to use: Hyper Ledger Fabric (HLF).
Developed under the Linux Foundation and IBM this network is a permissioned blockchain network for enterprise usage.
In contrast to public blockchains that allow unrestricted participation, here the participants must be well known and
trusted from every other participant.
This aligns with healthcare expectations, since the network does what was described previously, it complies with GDPR
and HIPAA and has a very configurable and modular architecture, giving flexibility to any organization.
Examples of usage for this could be record-keeping, drug tracing, management claims or contract storage. \cite{hyperledger} \cite{hyperledger-hippa-and-gdpr}

In addition, it is worth to mention that since integrity is the biggest concern for the use case in hands, an immutable ledger such
as the previously mentioned network would in this regard become handy.
Cryptographic linkage offers a significant advantage over traditional databases, since it guarantees that once recorded
data cannot be forged.
In case changed there is also a history for each piece of data, making audit trail friendly: patient records, medical histories,
prescriptions or supplements chain can be all securely managed. \cite{data-integrity-healthcare}

Also, one can argue that migrating from legacy infrastructures to such technology can be smooth: This framework is in such a degree modular that
composing everything becomes done quick and new functionalities can also be added with ease which still enables the corporations to innovate even more.
Moreover, it is important to mention that once implemented the stakeholders will cease to have borders between them: data dispersed across hospitals,
insurers, pharmaceutical firms, regulatory agencies and secure and efficient data sharing will no longer be a challenge, something that this thesis
focus in emphasize. \cite{multisectoral-healthcare}

Beyond these thematic pillars, the research also specifies use cases for HLF in healthcare: supply chain management, patient identification
and clinical trials.
On chain traceability would detect counterfeit drugs in supply chain and identity management capabilities would give the power to the patient
to control its data, thereby enhancing his privacy and reducing identity theft risks.

Besides that, this work also gives insights about the deployment of such network, particularly on-premise environments with the mentioned blockchain
framework.
Nevertheless, the focus is in reporting what is a kubernetes in doors deployment of such, providing the opportunity for evaluating its scalability,
security and performance, allowing scrutiny over the differences between conventional systems and these setups.
This technical assessment will further analyze advantages and drawbacks, exposing perspectives for organizations that intend to attend blockchain
adoption.
Potential future developments, current risks and areas of probable investigation will also be revealed.

The work aims to explore the application of such technology to the extent of facing the healthcare environment challenges.
Central questions are \textbf{“How can a private blockchain network be utilized
to improve data integrity, security, flexibility, and collaboration within the healthcare ecosystem?”} and \textbf{“What infrastructure design is necessary to support a
blockchain solution in such a vast and complex environment as healthcare?”}.
Through distributed ledgers, cryptographic algorithms and consensus protocols, Fabric can be used to provide a novel model of healthcare
data management: one that instigates trust among stakeholders, ensuring that sensitive information is tamper-proof and access-controlled.

In conclusion, there are challenges that must be faced and this research aims to expose those and possible solutions.
Moreover, network applications, technical architecture and strategic implications will be delved into and as healthcare continues
to evolve toward data-driven innovation, collaboration and trust, blockchain may emerge as bedrock for enforcing secure, efficient
and ethical management of patient information.


\subsection{Objectives and Finality}

If putted into deep thinking, in the sphere of the modern new world, blockchain technology has loom as an
auspicious solution to the given number of challenges faced by the healthcare industry.
It's enormous traits can range from cryptographic security, transparency and immutability and ,by retaining such characteristics,
it also seeks influence in securing and managing sensitive data.
It aims to safeguard the gathering of data across a networking of nodes, making it tamper-proof and erasing the traditional single points of failure.
In another deck of terms, blockchain can enable smooth data allocation between heterogeneous entities and yet making
certain that only conceded parties have privileges to specific and maybe confidential pieces of information. \cite{hyperledger-hippa-and-gdpr}

Within the vast array of blockchain frameworks, Hyperledger Fabric looms as particularly well-suited
to addressing the specific needs of the healthcare environment.
It provides the flexibility, scalability and security that serve as the genesis block for enterprise-level applications.
Unlike public blockchains such as Bitcoin or Ethereum, Hyperledger Fabric permissioned model allows more control over
access to the network, ensuring that only trusted participants are allowed to validate and access data.
This is a crucial feature in healthcare, where strict privacy regulations and the need for secure, authorized data
sharing are very important, not speaking about the need of on premise implementations which are also very possible
with this framework \cite{btc} \cite{hyperledger}.

Hyperledger Fabric's modular architecture also enables organizations to customize their networks to
meet specific needs.
It can be used for medical record management, pharmaceutical supply chain tracking, or patient identity verification, the platform can be adapted to a variety of use cases.
This flexibility not only facilitates innovation but also supports the integration of blockchain technology with existing legacy systems
which reduces the friction associated with technological adoption in healthcare settings. \cite{hyperledger-hippa-and-gdpr}

This investigation and the whole project emerges from the rising knowledge around the potential of the blockchain and other technologies
that are used currently.
With frequency, lots of people discuss the matter of decentralization, but lack on the sphere of trying to solve the problem in hands.
When there is a depth relation within the creation of a conjunction between centralized and more decentralized technologies, we
get the means to empower healthcare institutions to face their real problems.
While blockchain has already been explored in sectors such as finance and supply chain management, it's application in the healthcare sector remain
relatively fledgling.
Nevertheless, early trials and implementations have been showing promising in primrosing data security,
interoperability and collaboration between multiple organizations or even
stakeholders \cite{blockchain-in-finance} \cite{blockchain-in-supply-chain}.

The reliance from the healthcare on trust makes it an ideal candidate for the blockchain landscape, even
more if there is careful thinking about the need of securing and verifying sensitive patient data.
However,deploying such complex system in the healthcare context does not cease to be very challenging and demanding.
A lot of possible barriers to achieve this can be the Technical, regulatory, and operational spheres
of the sector.
Scalability is also another dual knife problem, and remains as so in large healthcare
environments with enormous and vast amounts of information and high transaction volumes.
This is a huge concern in both worlds since, within a blockchain there is the need of careful thinking such that there is the concern
regarding the degree of consensus which delays requests out.
Furthermore, the integration of blockchain with legacy systems and compliance with existing regulatory frameworks must be thoroughly evaluated. \cite{healthcare-it-trust} \cite{healthcare-blockchain-scalability}

This thesis is driven by a desire to explore these challenges and opportunities in depth.
Specifically,it seeks to understand how Hyperledger Fabric can be designed, implemented, and deployed to address the critical
needs of healthcare organizations.
By investigating the ways of deploying a private blockchain network within healthcare systems.

With this in mind, investigation questions that this work pretends to answer are the following:
\begin{enumerate}
\item \textbf{"How can a private blockchain network be utilized to improve data integrity, security, flexibility, and collaboration within the healthcare ecosystem?"}
\item \textbf{"What kind of infrastructure design is necessary to support a blockchain solution in such a vast and complex environment as healthcare?"}
\end{enumerate}

These questions are imposed, so that it becomes more evident in a qualitative way the achievement of the result
in mind.
Further assumptions about actual quantitative goals must be ensured as the thesis goes along but answering
this questions is the basis to come up with a suited solution in practical but also in theoretical knowledge terms.

As obvious as it may be, every research must have its own objectives and this one could not be different,
specially this one where there is the possibility to gather multiple objectives due to the countless numbers
of opportunities and challenges that could be faced regarding this procedure of recursion to the blockchain.
This becomes even more clear according to the area of investigation which is healthcare, an institution that
depending on the department becomes a high level risk sector to face, not only because of that but also mainly
because of it's many problems, it's public/private nature that is always changing, the regulations required
to be part of the procedures, the continuous innovation, the fact that is researched by a vast number of
different areas, the number of stakeholders involved, the number of organizations involved and the change
resistance.
With this in mind, the objectives covered will be the most general possible, and they will aim to
solve the majority of problems that the project may face and with this there are a lot of risks as well but those
will be also described further in this work.
The objectives that are intended to be reached in the realm of this
work are the following:

\subsection{Document Structure}
This dissertation is divided into several chapters: "Introduction,State of Art,Methodologies and Tools, Use
Cases,Discussion,Conclusions," and "References.".

\textbf {Introduction}: This chapter is reserved to give a contextualization and expose what is about
to be delved. This is accomplished by explaining the context, motivation, finality,objectives and document
structure.

\textbf{State of Art}: Within this chapter there is the supply of the most important theoretical concepts that
are pretty relevant for the case under scrutiny.

\textbf{Approach}: In this chapter, there is the explanation of DSR and how it is being used currently in general
terms, the explanation of SWOT analysis, the expected project plan, a list of risks, and the relevant tools
divided by area of concepts (blockchain, microservices, analytics and web development).

\textbf {Use Cases}: On this section, there will be mention of two major use cases that were conceived during the
project, standing as the biggest font of knowledge of this work. In each section, objectives, plans, discussions,
and even conclusions will be conducted. Also, one of the use cases is more practical, having topics that are more
specific to it's sphere, such as implementations, solution comparison, approach, and final architecture.

\textbf{Discussions}: This represents the discussion of the whole project and not a single use case. This is
 important to reflect over the results obtained as a whole and not only specific to a single use case.

\textbf {Conclusions}: In the conclusions section, there will be plenty of views around final considerations,
project limitations and future work. This will be done in the context of the general project, since every use
case already speaks about such things.

\textbf{References}: The references chapter will be around the bibliographical references that were
used along the dissertation.

\subsection{Answer Research Questions}\mbox{}\\
\textbf{Answer Research Questions} is the basis of the project, if answered this question will generate
practical and also theoretical insights that will foster innovation. The questions mentioned here, are the ones
that were acknowledged in the section 3, being those \textbf{”How can a private blockchain network be utilized to
improve data integrity, security, flexibility, and collaboration within the healthcare ecosystem?”}
(question 1). And \textbf{”What kind of infrastructure design is necessary to support a blockchain solution
in such a vast and complex environment as healthcare?”} (question 2), respectively. As it will be shown further
in this thesis, to gather the information to answer such answers, there will be mentioned the existence of 2 major
use cases: One works more as an extension, although inspected firstly and the second one, which is conducted using
DSR(Design Science Research), is what is required as genesis block for creating use cases that can then
be extended using the first use case. In other terms, the first one is mainly theoretical and the second one
try's to solve a real world problem within the healthcare realm. Answering those questions, not only is important
for the thesis itself but also for the healthcare sphere and other contexts that could adopt the same logic to
solve their problems, being essential answering them as such occurs where both practical and theoretical
inceptions are very important. Making the second project more appealing due to having proven results to achieve a
given solution.
Despite this being the main objective, inside the second use case there are others that are more quantitative,
thereby more concrete. Some can be used as metrics in other contexts, others may not but the importance of these
questions remain important although not that much specific when it comes to the refereed metrics and even the
requirements disposed in the use case section.

Both questions are answered by using the conjunction of both research prior use case and by delving into the
use case itself, although it is believed that the use cases represent better the potentialities of this work
because the theory by itself sometimes is not sufficient to cover vast cases, such the case of the sector under
study, more concretely, the healthcare environment.

\subsection{Contribute}\mbox{}\\
Another objective that stands out with honor is the honored contribution. Despite this being obvious
and also be part of the DSR methodology, it is still important to mention concretely this as an
objective.
